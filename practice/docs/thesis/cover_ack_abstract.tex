%!TEX root = ../main_v01.tex

\title{\large should probably decide whether to include chad's thing...}
%\fontsize{15}{17}\selectfont 
\author{Harriet Li}

\prevdegrees{B.S. Aerospace Engineering (2013)\\
             Massachusetts Institute of Technology}

\department{Department of Aeronautics and Astronautics}
% If the thesis is for two degrees simultaneously, list them both
% separated by \and like this:
% \degree{Doctor of Philosophy \and Master of Science}
\degree{Master of Science}
\degreemonth{June}
\degreeyear{2015}
\thesisdate{May 21, 2015}

%% By default, the thesis will be copyrighted to MIT.  If you need to copyright
%% the thesis to yourself, just specify the `vi' documentclass option.  If for
%% some reason you want to exactly specify the copyright notice text, you can
%% use the \copyrightnoticetext command.  
%\copyrightnoticetext{\copyright IBM, 1990.  Do not open till Xmas.}

% If there is more than one supervisor, use the \supervisor command
% once for each.
\supervisor{Karen Willcox}{Professor of Aeronautics and Astronautics}

% This is the department committee chairman, not the thesis committee
% chairman.  You should replace this with your Department's Committee
% Chairman.
\chairman{Paulo Lozano}{Associate Professor of Aeronautics and Astronautics\\
        Chair, Graduate Student Committee}


% Make the titlepage based on the above information.  If you need
% something special and can't use the standard form, you can specify
% the exact text of the titlepage yourself.  Put it in a titlepage
% environment and leave blank lines where you want vertical space.
% The spaces will be adjusted to fill the entire page.  The dotted
% lines for the signatures are made with the \signature command.
\maketitle

% The abstractpage environment sets up everything on the page except
% the text itself.  The title and other header material are put at the
% top of the page, and the supervisors are listed at the bottom.  A
% new page is begun both before and after.  Of course, an abstract may
% be more than one page itself.  If you need more control over the
% format of the page, you can use the abstract environment, which puts
% the word "Abstract" at the beginning and single spaces its text.

%% You can either \input (*not* \include) your abstract file, or you can put
%% the text of the abstract directly between the \begin{abstractpage} and
%% \end{abstractpage} commands.

% First copy: start a new page, and save the page number.
\cleardoublepage
% Uncomment the next line if you do NOT want a page number on your
% abstract and acknowledgments pages.
% \pagestyle{empty}
\setcounter{savepage}{\thepage}

%Uncomment the following three lines if you want an abstract page
\begin{abstractpage}
In scientific and engineering contexts, physical systems are represented by mathematical models, characterized by a set of parameters. The inverse problem arises when the parameters are unknown and one tries to infer these parameters based on observations. Solving the inverse problem can require many model simulations, which may be expensive for complex models; multiple models of varying fidelity and complexity may be available to describe the physical system. However, inferring the parameters may only be an intermediate step, and what is ultimately desired may be a low-dimensional Quantity of Interest (QoI); we refer to this as the goal-oriented inverse problem. We present a novel algorithm for solving the goal-oriented inverse problem, which allows one to manage the fidelity of modeling choices while solving the inverse problem.

We formulate a hierarchy of models, and assume that the QoI obtained by inferring the parameters with the highest-fidelity model is the most accurate QoI. We derive an estimate for the error in the QoI from inferring the parameters using a lower-fidelity model instead of the highest-fidelity model. This estimate can be localized to individual elements of a discretized domain, and this element-wise decomposition can then be used to adaptively form mixed-fidelity models. These mixed-fidelity models can be used to infer the parameters, while controlling the error in the QoI.

We demonstrate the method with two pairs of steady-state models in 2D. In one pair, the models differ in the physics included; in the other pair, the models differ in the space to which the parameters belong. In both cases, we are able to obtain a QoI estimate with a small relative error without having to solve the inverse problem with the high-fidelity model. We also demonstrate a case where solving the inverse problem with the high-fidelity model requires a more complex algorithm, but where our method gives a mixed-fidelity model with which we can infer parameters using a simple Newton solver, while achieving a low error in the QoI.


\end{abstractpage}

% Additional copy: start a new page, and reset the page number.  This way,
% the second copy of the abstract is not counted as separate pages.
% Uncomment the next 6 lines if you need two copies of the abstract
% page.
% \setcounter{page}{\thesavepage}
% \begin{abstractpage}
% In scientific and engineering contexts, physical systems are represented by mathematical models, characterized by a set of parameters. The inverse problem arises when the parameters are unknown and one tries to infer these parameters based on observations. Solving the inverse problem can require many model simulations, which may be expensive for complex models; multiple models of varying fidelity and complexity may be available to describe the physical system. However, inferring the parameters may only be an intermediate step, and what is ultimately desired may be a low-dimensional Quantity of Interest (QoI); we refer to this as the goal-oriented inverse problem. We present a novel algorithm for solving the goal-oriented inverse problem, which allows one to manage the fidelity of modeling choices while solving the inverse problem.

We formulate a hierarchy of models, and assume that the QoI obtained by inferring the parameters with the highest-fidelity model is the most accurate QoI. We derive an estimate for the error in the QoI from inferring the parameters using a lower-fidelity model instead of the highest-fidelity model. This estimate can be localized to individual elements of a discretized domain, and this element-wise decomposition can then be used to adaptively form mixed-fidelity models. These mixed-fidelity models can be used to infer the parameters, while controlling the error in the QoI.

We demonstrate the method with two pairs of steady-state models in 2D. In one pair, the models differ in the physics included; in the other pair, the models differ in the space to which the parameters belong. In both cases, we are able to obtain a QoI estimate with a small relative error without having to solve the inverse problem with the high-fidelity model. We also demonstrate a case where solving the inverse problem with the high-fidelity model requires a more complex algorithm, but where our method gives a mixed-fidelity model with which we can infer parameters using a simple Newton solver, while achieving a low error in the QoI.


% \end{abstractpage}

%Uncomment the following three command lines if you want Acknowledgements page
\cleardoublepage

\section*{Acknowledgments}

...
