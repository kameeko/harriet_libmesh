%!TEX root = ../main_v01.tex

\title{Model Adaptivity for Goal-Oriented Inference}
%\fontsize{15}{17}\selectfont 
\author{Harriet Li}

\prevdegrees{B.S. Aerospace Engineering (2013)\\
             Massachusetts Institute of Technology}

\department{Department of Aeronautics and Astronautics}
% If the thesis is for two degrees simultaneously, list them both
% separated by \and like this:
% \degree{Doctor of Philosophy \and Master of Science}
\degree{Master of Science}
\degreemonth{June}
\degreeyear{2015}
\thesisdate{May 21, 2015}

%% By default, the thesis will be copyrighted to MIT.  If you need to copyright
%% the thesis to yourself, just specify the `vi' documentclass option.  If for
%% some reason you want to exactly specify the copyright notice text, you can
%% use the \copyrightnoticetext command.  
%\copyrightnoticetext{\copyright IBM, 1990.  Do not open till Xmas.}

% If there is more than one supervisor, use the \supervisor command
% once for each.
\supervisor{Karen Willcox}{Professor of Aeronautics and Astronautics}

% This is the department committee chairman, not the thesis committee
% chairman.  You should replace this with your Department's Committee
% Chairman.
\chairman{Paulo Lozano}{Associate Professor of Aeronautics and Astronautics\\
        Chair, Graduate Student Committee}


% Make the titlepage based on the above information.  If you need
% something special and can't use the standard form, you can specify
% the exact text of the titlepage yourself.  Put it in a titlepage
% environment and leave blank lines where you want vertical space.
% The spaces will be adjusted to fill the entire page.  The dotted
% lines for the signatures are made with the \signature command.
\maketitle

% The abstractpage environment sets up everything on the page except
% the text itself.  The title and other header material are put at the
% top of the page, and the supervisors are listed at the bottom.  A
% new page is begun both before and after.  Of course, an abstract may
% be more than one page itself.  If you need more control over the
% format of the page, you can use the abstract environment, which puts
% the word "Abstract" at the beginning and single spaces its text.

%% You can either \input (*not* \include) your abstract file, or you can put
%% the text of the abstract directly between the \begin{abstractpage} and
%% \end{abstractpage} commands.

% First copy: start a new page, and save the page number.
\cleardoublepage
% Uncomment the next line if you do NOT want a page number on your
% abstract and acknowledgments pages.
% \pagestyle{empty}
\setcounter{savepage}{\thepage}

%Uncomment the following three lines if you want an abstract page
\begin{abstractpage}
Inference problems are often constrained by state equations, which arise from conservation laws, and constitutive relations. Often, a hierarchy of models can be derived from such laws and relations, reflecting varying fidelity versus cost tradeoffs. We introduce a goal oriented, model adaptive, inference algorithm, which allows users to achieve an arbitrary error tolerance, while minimizing the use of high fidelity constraints. Numerical experiments exercise the algorithm on a highly nonlinear inverse problem, and showcase the computational savings and robustness benefits of this new approach.

\end{abstractpage}

% Additional copy: start a new page, and reset the page number.  This way,
% the second copy of the abstract is not counted as separate pages.
% Uncomment the next 6 lines if you need two copies of the abstract
% page.
% \setcounter{page}{\thesavepage}
% \begin{abstractpage}
% Inference problems are often constrained by state equations, which arise from conservation laws, and constitutive relations. Often, a hierarchy of models can be derived from such laws and relations, reflecting varying fidelity versus cost tradeoffs. We introduce a goal oriented, model adaptive, inference algorithm, which allows users to achieve an arbitrary error tolerance, while minimizing the use of high fidelity constraints. Numerical experiments exercise the algorithm on a highly nonlinear inverse problem, and showcase the computational savings and robustness benefits of this new approach.

% \end{abstractpage}

%Uncomment the following three command lines if you want Acknowledgements page
\cleardoublepage

\section*{Acknowledgments}

This work would not have been possible without the help of many others, whom I hope I have, or will be able to, thank in more meaningful ways than I am able to here. 

I am most deeply grateful to my advisor, Professor Karen Willcox, for her support and guidance throughout my undergraduate years and in my graduate studies, for allowing me to explore the realm of research and computational methods through UROPs when I was curious but clueless, for suggesting new directions whenever I ran myself into a corner, and for helping me work around what I feared were fatal weak points in my self.

I am also very grateful to Dr. Vikram Garg, for allowing me to run with his ideas, for being patient in explaining new concepts, for describing these concepts in ways that were accessible to me even though my background was lacking, and for being the opposite of the scary person I was expecting when he was first introduced to me. 

I thank my parents, for doing their best to give me the freedom to pursue my interests, for being supportive of my choice to stay in school and conduct research even though I could no longer explain my work to them, and for forgiving me when I forgot their birthdays or to return their calls during the busiest times of the year. I thank my sister, for listening and reminding me of life outside of research.

I thank my friends, for keeping me sane, for tempting me to take breaks and keeping me company when I refused, for providing baked goods when I began to regret my dislike for coffee, for listening when I was upset and allowing me to reciprocate, for being interesting people with interesting stories and interesting personalities.

And I thank Dr. Bart van Bloemen Waanders, for allowing me to work with him over summers and explore areas of computational science outside of my thesis work.

This work was supported by the United States Department of Energy, Office of Advanced Scientific Computing Research (ASCR), Applied Mathematics Program, awards DE-FG02-08ER2585 and DE-SC0009297, as part of the DiaMonD Multifaceted Mathematics Integrated Capability Center.
