\section{Introduction}

Physical and engineering systems are often described using sophisticated mathematical models, such as coupled, nonlinear partial differential equations (PDEs). The inverse problem seeks to infer unknown model parameters using observed data; often these observations are limited, noisy and provide only indirect information about the unknown parameters. Many such inverse problems present a major computational challenge, since their solution involves repeated solutions of the mathematical model. In many cases, a given physical system can be represented with varying degrees of fidelity by different models. A higher-fidelity model more accurately represents reality, but is also usually more computationally expensive to solve. Lower-fidelity models can be solved inexpensively, at the expense of introducing approximations. In this work, we formulate an approach for leveraging multiple models of varying fidelities in solution of an inverse problem.

The choice of model should be informed by one's goal: for a range of applications, the parameters may be numerous, such as when corresponding to a discretized field, and yet what is ultimately of interest may be some low-dimensional quantity of interest (QoI). We consider this \textit{goal-oriented inverse problem}, where the goal of inferring parameters is to use them in predicting a QoI. In this context, one may choose a lower-fidelity model for the inference problem, sacrificing accuracy in the state and/or parameter estimates in exchange for reduced computational costs while keeping the error in the QoI to within some acceptable tolerance.  This paper formally poses the problem of managing the fidelity of modeling choices in solving the goal-oriented inverse problem, and develops an adjoint-based model adaptivity approach that achieves a desired level of accuracy in the QoI prediction. In particular, we adaptively form a mixed-fidelity model by using models of different levels of fidelity in different parts of the domain.

The simultaneous use of multiple models of varying fidelity for forward simulations is well established in some fields. Ref.~\cite{Liuetal03} categorizes two main strategies for combining models: hierarchical and concurrent methods. Hierarchical methods (also known as information-passing or sequential methods) take the results of a simulation using the high-fidelity model and use them to inform a lower-fidelity model that is used globally. An example application is the modeling of the molecular structure of a material to determine parameters for constitutive equations~\cite{Haoetal03,Weietal04}. In contrast, concurrent methods (also called hybrid methods) simultaneously solve the higher- and lower-fidelity models in different parts of the domain. Applications include computational mechanics~\cite{Khareetal08,Prudetal08} and fluid dynamics~\cite{AlexGarTar02,FatGerQua01,Garcetal99,LucKinBer02,vanOpstaletal15,WadErw90}.
% KW: I feel that we may be missing some important literature here, e.g., the work of Alexander, Tartakovsky
We focus on concurrent methods of combining models, which have desirable features: in the case where the high-fidelity model is nonlinear, replacing it with a linear lower-fidelity model in most of the domain can reduce the number of iterative solves needed; when the high-fidelity model has a fine resolution and/or many parameters, replacing it with a lower-fidelity model can reduce the number of degrees of freedom of the mixed-fidelity model.

Goal-oriented approaches prioritize accuracy in the QoI over accuracy in the states and/or parameters; one aims to reduce the cost of solving the problem (whether forward or inverse) while maintaining an acceptably low error tolerance for the QoI. In the context of the forward problem, goal-oriented methods have been developed for mesh refinement \cite{BecRann01,PrudOden99,VendDarm00,Yano12} and for model adaptivity~\cite{BraackErn03,OdenPrudetal06}. These methods derive an adjoint-based a posteriori error estimate for an output functional, and this estimate is used to guide adaptive mesh and/or model refinement.

The inverse problem setting has also seen some work on goal-oriented methods. To begin with, it has been recognized that for the goal-oriented inverse problem, fully resolving the parameters is often unnecessary to accurately compute the QoI. For example, for a discretized linear inverse problem, one can find a low-dimensional subspace of the parameter space that is both informed by observations and informative to the QoI~\cite{LiebWill13}. This subspace provides a low-dimensional map from the observations directly to the QoI, sacrificing accuracy in the inferred parameters for accuracy in the QoI that is computed from them. For the linear Gaussian Bayesian inverse problem, one can calculate an optimal approximation to the predictive posterior of the QoI without fully calculating the posterior distribution of the parameters~\cite{Span16}. Mesh refinement in the goal-oriented inverse problem is addressed in~\cite{BecVex05}; in that work, Becker and Vexler derive an a posteriori estimate of the error in the QoI caused by discretizing the infinite-dimensional inverse problem, and this error estimate is used to adaptively refine the mesh. Also, an outline extending the approach to goal-oriented model adaptivity described in~\cite{OdenPrudetal06} to the inverse problem is presented in~\cite{OdenPrudetal10}. 

In this paper, we target model adaptivity in the solution of goal-oriented inverse problems. We present a robust, adaptive framework that identifies the subregions of the domain where high-fidelity representation of physics and parameters is important to achieving a desired accuracy in the QoI. To achieve this, we combine ideas from goal-oriented model adaptivity for forward problems and goal-oriented methods for inverse problems. In particular, our method allows one to systematically manage the use of multiple models in the context of the goal-oriented inverse problem, so as to minimize the error in a QoI prediction. Taking the inverse problem with the highest-fidelity model as our reference QoI prediction, we derive a third-order estimate for the QoI error from using a lower-fidelity model. This estimate can be localized, and the error decomposition then used to guide the formation of mixed-fidelity models with which to solve the inverse problem, while minimizing the error in the QoI. 
%We note that while the approach outlined in~\cite{OdenPrudetal10} could be applied to the same task, one main difference is that we compute additional quantities to obtain a higher-order QoI error estimate.

The remainder of this paper is organized as follows, \Cref{sec:form} presents the mathematical formulation and error analysis for the multi-model inverse problem. \Cref{sec:alg} discusses the goal oriented inference algorithm and presents a computational complexity analysis. \Cref{sec:numexp} shows the application of this algorithm to a model problem and a contaminant flow problem. Finally, we present conclusions and directions for future work in \Cref{sec:conc}.
