An inverse problem seeks to infer unknown model parameters using observed data. We consider a \textit{goal-oriented inverse problem}, where the goal of inferring parameters is to use them in predicting a quantity of interest (QoI). Recognizing that multiple models of varying fidelity and computational cost may be available to describe the physical system, we formulate a goal-oriented model adaptivity approach that leverages multiple models while controlling the error in the QoI prediction. In particular, we adaptively form a mixed-fidelity model by using models of different levels of fidelity in different subregions of the domain. Taking the solution of the inverse problem with the highest-fidelity model as our reference QoI prediction, we derive an adjoint-based third-order estimate for the QoI error from using a lower-fidelity model. Localization of this error then guides the formation of mixed-fidelity models. We demonstrate the method for example problems described by convection-diffusion-reaction models. For these examples, our mixed-fidelity models use the high-fidelity model over only a small portion of the domain, but result in QoI estimates with small relative errors. We also demonstrate that the mixed-fidelity inverse problems can be cheaper to solve and less sensitive to the initial guess than the high-fidelity inverse problems.

