\section{Conclusion}\label{sec:conc}

We adaptively create mixed-fidelity models to solve goal-oriented inverse problems. The paper develops an error estimator that drives the adaptation, so as to minimize the error in the QoI calculated from the inferred parameters. We applied this method to pairs of low- and high-fidelity models of convection-diffusion-reaction phenomena. The results showed QoI estimates with a small relative error even when the high-fidelity model was used only in a small portion of the domain. In these cases, the localization of the error estimate also indicated regions of the domain that were important to the interactions between the observations and the QoI. A direction for extension of this work is to the case of the statistical inverse problem.  One way we could potentially apply this work to the statistical inverse problem is by reducing the parameter space that needs to be sampled. Such a direction is suggested by the results presented in \Cref{sec:diffvcdr3D}, where the mixed-fidelity model had significantly fewer degrees of freedom in its parameter field than the high-fidelity model, and thus a smaller parameter space. One could explore creation of an alternative statistical inverse problem that, by utilizing a mixed-fidelity model with fewer degrees of freedom in its parameter field, requires exploration of a small parameter space with minimal compromise in the predictive posterior.

Another potential approach would be to extend our method to the creation of mixed-fidelity models that are used as surrogates; these surrogate models can be evaluated in place of the high-fidelity model, thus decoupling the number of expensive forward evaluations of the high-fidelity model needed from the number of posterior parameter distribution samples that is desired \cite{Con14}. In such a case, the cost of creating the mixed-fidelity model would be amortized over a large number of posterior samples using the cheaper mixed-fidelity model in place of the high-fidelity model.
