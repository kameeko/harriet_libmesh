\documentclass[12pt, letterpaper]{article}
\usepackage{amsfonts}
\usepackage{amsmath,amsthm,amssymb,amsfonts,amsbsy,latexsym}
\usepackage{color}
\usepackage{changepage}
\usepackage{parskip}
\usepackage[margin=0.75in]{geometry}

\newcommand{\answer}[1]{\begin{adjustwidth}{1cm}{}{\color{blue}#1}\end{adjustwidth}}
\newcommand{\red}[1]{{\color{red}#1}}
\newcommand{\blue}[1]{{\color{blue}#1}}
\newcommand{\green}[1]{{\color{green}#1}}

\newcommand{\notdone}{{\color{red}{Changes not yet made.}}}
\newcommand{\done}{{\color{green}{Changes made.}}}

\begin{document}

\section*{Response to Referee Comments}

We thank the reviewers for their detailed and insightful comments and helping us strengthen the manuscript. Our point-by-point responses to the critique are outlined below, with reviewer comments in black and \blue{responses in blue}. 

%%%%%%%%%%%%%%%%%%%%%%%%%%%%%%%%%%%%%%%%%%%%%%%%%%
%%%%%%%%%%%%%%%%%%%%%%%%%%%%%%%%%%%%%%%%%%%%%%%%%%

\subsection*{Reviewer 2}

I had three major concerns with the previous version.

My first and last concern have been dealt with.

However, my second concern (on the degree of compatibility) is not dealt with adequately (in particular, Proposition 1 is not true in general). I therefore recommend that the authors re-submit an updated version for a subsequent review. My concern is further outlined below.

Although the part below Eqs (4a) and (4b) contains some explanation regarding compatibility of $U_{HF}$ and $U_{LF}$, this is not made mathematically explicit.
Therefore the result (7) of Proposition 1 is not necessarily true. For example, there is no reason why a low-fidelity solution can be substituted in a high-fidelity (bi)linear form, as done in (7). This requires suitable compatibility of the low-fidelity space with the high-fidelity space that has not been made mathematically explicit (some inclusion is actually assumed). In the proof, for example, eq. (9) is not necessarily true.
I have two recommendations: 1. restrict the assumption of different spaces to it being the same space, or 2. introduce suitable required inclusions for these spaces. 

\answer{We added a formal definition of compatibility just before Proposition 1.}

\end{document}
