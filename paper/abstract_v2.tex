In scientific and engineering contexts, physical systems are represented by mathematical models, characterized by a set of parameters. The inverse problem arises when the parameters are unknown and one tries to infer these parameters based on observations. Solving the inverse problem can require many model simulations, which may be expensive for complex models; multiple models of varying fidelity and complexity may be available to describe the physical system. In the case of the \textit{goal-oriented inverse problem}, the goal of inferring parameters is to use them in predicting a Quantity of Interest (QoI). We present a novel adaptive framework for the solution of goal-oriented inverse problems, which can identify the regions of the domain where high-fidelity representation of the physics and the parameter and state space is important to achieving a desired accuracy in the QoI. Taking the inverse problem with the highest-fidelity model as our reference QoI prediction, we derive a third-order estimate for the QoI error from using a lower-fidelity model. This estimate can be localized, and the error decomposition then used to guide the formation of mixed-fidelity models with which to solve the inverse problem, while minimizing the error in the QoI. We demonstrate the method with pairs of steady-state convection-diffusion-reaction models. We are able to obtain a QoI estimate with a small relative error without having to solve the inverse problem with the high-fidelity model. We show that the mixed-fidelity inverse problems can be cheaper to solve and less sensitive to initial guess than the high-fidelity inverse problems.

%from karen about v1: 
%"minimizing the use of high-fidelity constraints" -> won't make sense
%needs to be expanded + 1-2 sentences on approach; sentence on results needs to be more specific
