\section{Introduction}

Parameter estimation/inference is an important component of mathematical modeling. The ability to infer parameters via an inverse problem allows practitioners to gain important information about systems, processes and the models that are used to describe them. Such inference also avoids intrusive experiments, which may otherwise be prohibitively expensive or excessively intrusive. Once parameters have been inferred via an inverse problem, they can be used for forward propogation of the model and computing relevant goals and quantities of interest for the modeling/decision making process.

Data, constraints and prior information form the critical components of an inverse problem formulation. There is an extensive literature pertaining to all these components, pertaining to the ascription/filtering of noise in data, the choice of regularization/prior information, the efficient incorporation of constraints in solving the problem and so on. Specifically, the work on the constraints or 'state equations' in the inverse problem focuses on methods to reduce the computational expense of solving/inverting the models that represent them. Such work includes the use of reduced order models to represent constraints, adaptive mesh refinement to control the size of the state equations among other strategies.

Models that represent phenomena and processes can usually be organized in a hierarchy according to their fidelity and expense. The use of multiple models simultaneously for forward simulation is well established. In this work, we extend the use of the multi-fidelity models to inverse problems. We present a systematic method to use different constraints in different regions of the domain, while controlling the error due to the use of lower fidelity models in a target quantity of interest.

The rest of the paper is organized as follows, Section~\ref{} presents the mathematical formulation and error analysis for the multi-model inverse problem. Then, in Section~\ref{} we discuss the goal oriented inference algorithm and its application to a model problem and a contaminant flow problem. Finally, we present conclusions and direction for future work in Section~\ref{}.
