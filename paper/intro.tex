\section{Introduction}

Parameter estimation/inference is an important component of mathematical modeling. The ability to infer parameters via an inverse problem allows practitioners to gain important information about systems, processes and the models that are used to describe them. Such inference also avoids intrusive experiments, which may otherwise be prohibitively expensive or impractical. Once parameters have been inferred via an inverse problem, they can be used for forward propagation of the model and, computing relevant quantities of interest for the modeling/decision making process.

Data, constraints and prior information form the critical components of an inverse problem formulation. There is an extensive literature pertaining to all these components, including the filtering of noise in data~\cite{}, the choice of regularization or representation of prior information~\cite{}, the efficient incorporation of constraints in solving the problem~\cite{}, and so on. Specifically, the work on the constraints defined by `state equations' in the inverse problem focuses on reducing the computational expense of solving the models that represent them. Such work includes the use of reduced order models to represent constraints~\cite{} and adaptive mesh refinement to control the size of the state equation discretizations~\cite{BecVex05,}, among other strategies.

Models that represent phenomena and processes can usually be organized in a hierarchy according to their fidelity and/or expense. The use of multiple models simultaneously for forward simulation is well established. Two main strategies exist for combining models: hierarchical and concurrent methods. Hierarchical methods (also known as information-passing or sequential methods) take the results of a simulation using the high-fidelity model and use them to inform a lower-fidelity model that is used globally (for example, modeling the molecular structure of a material to determine parameters for constitutive equations~\cite{Haoetal03}). In this work, we focus on multi-fidelity models formed using concurrent methods, which simultaneously solve the higher- and lower-fidelity models in different parts of the domain. For example, in~\cite{Khareetal08} atomistic models capable of describing bond-breaking behaviors are applied to small clusters of atoms in regions relevant to the formation of fractures, while a continuum model is applied in the rest of the domain. Concurrent methods are also applied in \cite{vanOpstaletal15} to combine the linear Stokes equation (low-fidelity) and the nonlinear Navier-Stokes equation (high-fidelity) based on an a posteriori estimate of the error in a QoI, and in \cite{LucKinBer02} to combine different reduced-order models, depending on the presence of shocks.

%in \cite{AntHein10}, balanced truncation model reduction is applied to the part of the domain outside of where the optimization variables are located in order to reduce the overall cost of a forward solve. 
%The manner in which the different models are interfaced is problem-dependent; in \cite{Abra98, AlexGarTar02}, a ``handshake'' region is used to couple concurrent particle and continuum models, and in \cite{AlexGarTar02}, interface coupling models are introduced to reconcile uncertainties in the different models.

In this work, we extend the use of multi-fidelity models to inverse problems. Specifically, we consider the case where there is a target quantity of interest to be computed from the inferred parameters. The use of concurrent multi-fidelity models for the goal-oriented forward problem is described in \cite{OdenPrudetal06}, a method using adjoints for the goal-oriented forward problem is described; an extension to inverse problems is described in \cite{OdenPrudetal10}. We present a systematic method to use different constraints in different regions of the domain, while controlling the error due to the use of lower fidelity models, in a target quantity of interest. Our approach is based on a novel use of adjoint error estimation, which enables the computation of third-order QoI error estimates, at the cost of solving an additional adjoint problem.

The rest of the paper is organized as follows. \Cref{sect:form} presents the mathematical formulation and error analysis for the multi-model inverse problem. Then, in \Cref{sect:alg} we discuss the goal-oriented inference algorithm and present a computational complexity analysis. \Cref{sect:numexp} shows the application of this algorithm to a model problem and a contaminant flow problem. Finally, we present conclusions and directions for future work in \Cref{sect:conc}.

