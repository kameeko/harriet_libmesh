\section{Introduction}
%
Physical and engineering systems are often described using sophisticated mathematical models, such as coupled, nonlinear partial differential equations. Recent years have seen important advances in the accurate and reliable solution of the `forward problem', i.e.\ the solution of such complex models given appropriate initial/boundary conditions and constitutive parameters. In particular, adaptive methods have emerged as a highly effective paradigm for solving the forward problem. `Goal-oriented error estimation' algorithms are an important subset of adaptive methods, where a computational grid or even the model itself is adapted to obtain the desired accuracy in a particular goal (QoI), while minimizing the number of grid points or model complexity.

Often, the solution of the forward problem is embedded in a larger framework, where our objective is to either infer unknown model parameters (inverse problems) or the selection of optimal parameters to minimize or maximize a desired quantity (optimization problems) \cite{Taran05, BanksKuhn89}. Such inverse problems present a computational major challenge, since they often involve repeated solutions of the forward problem, and contain additional equations that arise from their mathematical formulation. Adaptive methods have been applied to make inverse problems tractable as well, including work on the \textit{goal-oriented inverse problem}. In a goal-oriented inverse problem, the emphasis is on identifying a low dimensional subset of the parameter or state space, which is both informed by the data, and has an impact on the goal functional. Various algorithms have been presented in the literature for solving the goal-oriented inverse problem, however, there has been little work in the use of adaptive modeling in such a setting.

In this work, we present a robust, adaptive framework for the solution of goal-oriented inverse problems, which can identify the appropriate subset of the parameter, state space or model features/complexity to achieve a desired accuracy in the QoI. In doing so, we aim to combine the ideas from goal-oriented model adaptivity for forward problems, and goal-oriented methods for inverse problems. In particular, our method that allows one to systematically manage the use of multiple models, in an inverse problem, so as to minimize the error in a QoI prediction. Taking the inverse problem with the highest-fidelity model as our reference QoI prediction, we derive a third-order estimate for the QoI error from using a lower-fidelity model. This estimate can be localized, and the error decomposition then used to guide the formation of mixed-fidelity models with which to solve the inverse problem, while minimizing the error in the QoI.

% Goal-oriented approaches prioritize accuracy in the QoI over accuracy in the states and/or parameters. One way to control the cost of the model is to consider different meshes. In the context of the forward problem, so and so have done a-b-c. In the context of the inverse problem, so and so have done x-y-z.

% Another way to control the cost of the model is to change the fidelity to which the physics are represented. The simultaneous use of multifidelity models for forward simulations is well established; two main strategies exist for combining models: hierarchical and concurrent methods. (Describe what these mean)

% We focus on current methods of combining models. concurrent multifidelity models appear in a range of contexts, such as...

The rest of the paper is organized as follows, \Cref{sec:form} presents the mathematical formulation and error analysis for the multi-model inverse problem. Then, in \Cref{sec:alg} we discuss the goal oriented inference algorithm and present a computational complexity analysis. \Cref{sec:numexp} shows the application of this algorithm to a model problem and a contaminant flow problem. Finally, we present conclusions and directions for future work in \Cref{sec:conc}.
