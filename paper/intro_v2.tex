\section{Introduction}

Physical and engineering systems are often described using sophisticated mathematical models, such as coupled, nonlinear partial differential equations (PDEs). In the inverse problem, the parameters of the model are unknown and one tries to infer their values based on observations of the states or parameters~\cite{Taran05, BanksKuhn89}. Such inverse problems present a major computational challenge, since they often involve repeated solutions of the forward problem. A given physical system can be represented with varying degrees of fidelity by different models; a higher-fidelity model is able to more accurately represent reality, but is also usually more difficult to solve. The choice of model can be informed by one's goal; for a range of applications, the parameters may be numerous, such as when corresponding to a discretized field, and yet what is ultimately of interest may be some low-dimensional Quantity of Interest (QoI). In the context of the \textit{goal-oriented inverse problem}, where the goal of inferring parameters is to use them in predicting a QoI, one may choose a lower-fidelity model with which to perform inference, sacrificing accuracy in the state and/or parameters in exchange for reduced computational costs while keeping the error in the QoI to within some acceptable tolerance.  In this work, we present an approach to manage the fidelity of modeling choices in solving the inverse problem, so as to achieve a desired level of accuracy in the QoI prediction, without necessarily resolving the parameters accurately. We adaptively form a mixed-fidelity model with which we solve the inverse problem by using models of different levels of fidelity in different parts of the domain.

The simultaneous use of multiple models of varying fidelity for forward simulations is well established; two main strategies exist for combining models: hierarchical and concurrent methods~\cite{Liuetal03}. Hierarchical methods (also known as information-passing or sequential methods) take the results of a simulation using the high-fidelity model and use them to inform a lower-fidelity model that is used globally. A common application is the modeling of the molecular structure of a material to determine parameters for constitutive equations~\cite{Haoetal03,Weietal04}. In contrast, concurrent methods simultaneously solve the higher- and lower-fidelity models in different parts of the domain. Applications include computational mechanics~\cite{Khareetal08,Prudetal08} and fluid dynamics~\cite{vanOpstaletal15,LucKinBer02,FatGerQua01}. We focus on concurrent methods of combining models, which have desirable features: in the case where the high-fidelity model is nonlinear, replacing it with a linear lower-fidelity model in most of the domain can reduce the need for iterative solves; when the high-fidelity model has a fine resolution and/or many parameters, replacing it with a lower-fidelity model can reduce the number of degrees of freedom of the mixed-fidelity model.

The best way to choose the subdomains for the different models depends of course on what one wishes to achieve. Goal-oriented approaches prioritize accuracy in the QoI over accuracy in the states and/or parameters; one aims to reduce the cost of solving the problem (whether forward or inverse) while maintaining acceptably low error estimate of the QoI. The problem of mesh-refinement can be thought of as one of combining models corresponding to different discretizations of a continuous equation. In the context of the forward problem, methods for goal-oriented mesh-refinement using adjoints are described in~\cite{PrudOden99,VendDarm00,BecRann01}; an a posteriori error estimate in an output functional is derived, and this estimate is used to guide adaptive mesh-refinement. A framework for automated mesh-refinement to calculate a QoI to a prescribed accuracy is described in~\cite{Yano12}. Goal-oriented model adaptivity aims to optimally mix models differing in their mesh resolution and/or physics included~\cite{OdenPrudetal06,BraackErn03}.

Work has also been done on goal-oriented methods for the inverse problem. Mesh-refinement in the goal-oriented inverse problem is addressed in~\cite{BecVex05}; in this work, Becker and Vexler derive an a posteriori estimate of the error in the QoI caused by discretizing the infinite dimensional inverse problem, and this error estimate is used to adaptively refine the mesh. An outline extending the approach to goal-oriented model adaptivity described in~\cite{OdenPrudetal06} to the inverse problem is presented in~\cite{OdenPrudetal10}. In the goal-oriented inverse problem, the data is often insufficient to fully resolve the inferred parameters; in addition, fully resolving these parameters is unnecessary to accurately computing the QoI. This idea is formalixed in ~\cite{LiebWill13}; for a discretized linear inverse problem, one can find a low-dimensional subspace of the parameter space that is both informed by observations and informative to the QoI. This subspace is used to produce a low-dimensional map from the observations directly to the QoI, sacrificing accuracy in the inferred parameters for accuracy in the QoI that is computed from them. For the linear Gaussian Bayesian inverse problem, one can calculate an optimal approximation to the predictive posterior of the QoI without fully calculating the posterior distribution of the parameters~\cite{Span16}. 

In this work, we present a robust, adaptive framework for the solution of goal-oriented inverse problems, which can identify the regions of the domain where high-fidelity representation of the physics and the parameter and state space is important to achieving a desired accuracy in the QoI. In doing so, we combine ideas from goal-oriented model adaptivity for forward problems, and goal-oriented methods for inverse problems. In particular, our method allows one to systematically manage the use of multiple models in the context of the goal-oriented inverse problem, so as to minimize the error in a QoI prediction. Taking the inverse problem with the highest-fidelity model as our reference QoI prediction, we derive a third-order estimate for the QoI error from using a lower-fidelity model. This estimate can be localized, and the error decomposition then used to guide the formation of mixed-fidelity models with which to solve the inverse problem, while minimizing the error in the QoI. We note that while the approach outlined in~\cite{OdenPrudetal10} could be applied to the same task, one main difference is that we compute additional quantities to obtain a higher-order QoI error estimate.

The rest of the paper is organized as follows, \Cref{sec:form} presents the mathematical formulation and error analysis for the multi-model inverse problem. Then, in \Cref{sec:alg} we discuss the goal oriented inference algorithm and present a computational complexity analysis. \Cref{sec:numexp} shows the application of this algorithm to a model problem and a contaminant flow problem. Finally, we present conclusions and directions for future work in \Cref{sec:conc}.
