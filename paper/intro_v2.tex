\section{Introduction}

In scientific and engineering contexts, physical systems are represented by mathematical models, which often take the form of partial differential equations (PDEs). These models are characterized by a set of parameters, and relate these parameters to states that can be observed. In the inverse problem, the parameters are unknown and one tries to infer their values based on observations of the states or parameters \cite{Taran05, BanksKuhn89}. For a range of applications, the parameters may be numerous, and yet what is ultimately of interest may be some low-dimensional Quantity of Interest (QoI); we refer to the setting where the goal of inferring parameters is to use them in predicting a QoI as a \textit{goal-oriented inverse problem}. A given physical system can be represented with varying degrees of fidelity by different models; a higher-fidelity model is able to more accurately represent reality, but is also usually more difficult to solve. In the context of the goal-oriented inverse problem, one may choose a lower-fidelity model with which to perform inference, sacrificing accuracy in the state and/or parameters in exchange for reduced computational costs while keeping the error in the QoI to within some acceptable tolerance. In this work, we present an approach to manage the fidelity of modeling choices in solving the inverse problem, so as to achieve a desired level of accuracy in the QoI prediction, without necessarily resolving the parameters accurately. We adaptively form a mixed-fidelity model with which we solve the inverse problem by using models of different levels of fidelity in different parts of the domain.

% Goal-oriented approaches prioritize accuracy in the QoI over accuracy in the states and/or parameters. One way to control the cost of the model is to consider different meshes. In the context of the forward problem, so and so have done a-b-c. In the context of the inverse problem, so and so have done x-y-z.

% Another way to control the cost of the model is to change the fidelity to which the physics are represented. The simultaneous use of multifidelity models for forward simulations is well established; two main strategies exist for combining models: hierarchical and concurrent methods. (Describe what these mean)

% We focus on current methods of combining models. concurrent multifidelity models appear in a range of contexts, such as...

We aim to combine the ideas in goal-oriented methods for inverse problems, and goal-oriented model adaptivity approaches for forward modeling. The objective of this work is to formulate a method that allows one to systematically manage the use of multiple models in the context of the goal-oriented inverse problem, so as to minimize the error in a QoI prediction. To do this, we first assume that solving the inverse problem with the highest-fidelity model would result in the most accurate QoI, but that solving this inverse problem is prohibitively expensive. We derive a third-order estimate for the QoI error from inferring the parameters using a lower-fidelity model. This estimate can be localized, and the error decomposition then used to guide the formation of mixed-fidelity models with which to solve the inverse problem, while minimizing the error in the QoI. 

The rest of the paper is organized as follows, \Cref{sec:form} presents the mathematical formulation and error analysis for the multi-model inverse problem. Then, in \Cref{sec:alg} we discuss the goal oriented inference algorithm and present a computational complexity analysis. \Cref{sec:numexp} shows the application of this algorithm to a model problem and a contaminant flow problem. Finally, we present conclusions and directions for future work in \Cref{sec:conc}.
