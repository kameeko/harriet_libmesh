\section{Conclusion}\label{sect:conc}

We have presented an error estimator that can be used to adaptively create a mixed-fidelity model with which to solve a goal-oriented inverse problem, so as to minimize the error in the QoI calculated from the inferred parameters. We applied this method to pairs of low- and high-fidelity models of convection-diffusion-reaction phenomena and were able to obtain QoI estimates with a small relative error without having to solve the inverse problem with the high-fidelity model. In these cases, the localization of the error estimate also indicated regions of the domain that were important to the interaction of the observations and the QoI.

%------------------------------------------------------------%
\section{Future Work} 
%------------------------------------------------------------%

A direction for extension of this work is to the case of the statistical inverse problem. Thus far in this work, we have considered the deterministic inverse problem, as described in \Cref{sec:setup}; we seek to infer the parameter values that optimally fit the observations and the prior beliefs embedded in the regularization. However, we can rarely, if ever, be certain that the inferred values are correct, whether this be due to epistemic uncertainty from a lack of knowledge or aleatoric uncertainty from inherent variability in the physical system, or both \cite{Ober04}. One may attempt to capture the uncertainty in the inferred parameters by representing them as random variables with a probability distribution; inferring the distribution of the parameters given some observations is the statistical inverse problem. 

The statistical inverse problem is often solved in a Bayesian framework. Bayes' rule is used to combine a prior distribution, which captures prior beliefs about the parameters, and a likelihood distribution, which captures the likelihood of observations given an instance of the parameter values and a model of noise in the observations, to give a posterior distribution on the parameters. Since there is generally no analytical expression for this posterior distribution, it is usually characterized by samples from the distribution. Sampling methods like the widely-used Markov chain Monte Carlo (MCMC) method require many evaluations of the forward model, and since the number of samples needed grows exponentially with the dimension of the parameter space, this problem becomes intractable for large parameter spaces. In engineering contexts, it is still usually the case, however, that we are ultimately interested in a low-dimensional QoI, and it is the uncertainty in this low-dimensional quantity that we wish to capture; this low-dimensional distribution is referred to as the predictive posterior.

One way we could potentially apply this work to the statistical inverse problem is by reducing the parameter space that needs to be sampled. Such a direction is suggested by the results presented in \Cref{sec:diffvcdr3D}, where the mixed-fidelity model had significantly fewer degrees of freedom in its parameter field than the high-fidelity model, and thus a smaller parameter space. In the case of a linear model and observation operator with a Gaussian prior and additive Gaussian noise, there are parallels between the objective function of the deterministic inverse problem with Tikhonov regularization and the mode of the posterior distribution. In \cite{Martetal12}, a method is described for creating proposal distributions, drawing from these parallels; both the linear Gaussian and nonlinear cases are addressed. Similarly, these parallels could potentially be drawn upon to extend this work to the creation of an alternative statistical inverse problem that, by utilizing a mixed-fidelity model with fewer degrees of freedom in its parameter field, requires exploration of a small parameter space with minimal compromise in the predictive posterior.

Another potential approach would be to extend our method to the creation of mixed-fidelity models that are used as surrogates; these surrogate models can be evaluated in place of the high-fidelity model, thus decoupling the number of expensive forward evaluations of the high-fidelity model needed from the number of posterior parameter distribution samples that is desired \cite{Con14}. The samples obtained using such a surrogate might sacrifice accuracy in representing the posterior parameter distribution for accuracy in representing the predictive posterior distribution. 

%other directions for extensions that I can't seem to fit in:
%mixing models in time as well? (different mixes of models at different time steps?)
%superadj on intermediate mesh; preliminary results suggest that error estimates remain reasonable and generation of mixed-fidelity models can continue despite the additional approximations to the supplementary adjoint 

%\item \Cref{alg:refSeries} is also ammenable to an offline-online decomposition, analagous to that proposed in \cite{LiebWill13}. In the case where both the low-fidelity and high-fidelity inverse problems are linear in the data, and the QoI is linear in the state and parameters, one may first compute and store the supplementary adjoint $\Lambda_0$ for the low-fidelity model. As new data is received, one can compute the $\Psi_{LF}$ for this new data; evaluating \cref{eq:finErrExp} with the stored $\Lambda_0$ and the new $\Psi_{LF}$ gives an exact estimate of the error in the QoI, and thus effectively the high-fidelity QoI (NOT TRUE the primary and aux vars appear in the rhs...). When these linearities do not all hold, one cannot obtain an exact error estimate. The offline phase would consist of adaptively creating a mixed-fidelity model with an appropriate error tolerance given some expected observations $d^*$, and storing its suppementary adjoint $\Lambda_{MF}^*$. As new data is received, one can compute the $\Psi_{MF}$ for the mixed-fidelity model and the new data; evaluating \cref{eq:finErrExp} with the stored $\Lambda_{MF}^*$ and the new $\Psi_{MF}$ gives an estimate of the error in the QoI, and thus an effective estimate of the high-fidelity QoI. 

