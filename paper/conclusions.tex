\section{Conclusion}\label{sect:conc}
Future work:
\begin{itemize}
\item stochastic?
\item mixing models in time as well? (different mixes of models at different time steps?)
\end{itemize}

Extensions:
\begin{itemize}
\item superadj on intermediary mesh
\item \Cref{alg:refSeries} is also ammenable to an offline-online decomposition, analagous to that proposed in \cite{LiebWill13}. In the case where both the low-fidelity and high-fidelity inverse problems are linear in the data, and the QoI is linear in the state and parameters, one may first compute and store the supplementary adjoint $\Lambda_0$ for the low-fidelity model. As new data is received, one can compute the $\Psi_{LF}$ for this new data; evaluating \cref{eq:finErrExp} with the stored $\Lambda_0$ and the new $\Psi_{LF}$ gives an exact estimate of the error in the QoI, and thus effectively the high-fidelity QoI. When these linearities do not all hold, one cannot obtain an exact error estimate. The offline phase would consist of adaptively creating a mixed-fidelity model with an appropriate error tolerance given some expected observations $d^*$, and storing its suppementary adjoint $\Lambda_{MF}^*$. As new data is received, one can compute the $\Psi_{MF}$ for the mixed-fidelity model and the new data; evaluating \cref{eq:finErrExp} with the stored $\Lambda_{MF}^*$ and the new $\Psi_{MF}$ gives an estimate of the error in the QoI, and thus an effective estimate of the high-fidelity QoI. 
\end{itemize}

note that things can get weird if adapted and online data are very different
